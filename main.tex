\documentclass[a4paper,titlepage]{book}
\usepackage[utf8]{inputenc}
\usepackage[italian]{babel}

\usepackage{natbib}
\usepackage{graphicx}
\usepackage{hyperref}
\usepackage{frontespizio}

\newcommand{\overleaf}{Overleaf}

\begin{document}

% \maketitle
\begin{frontespizio}
    % Qui verrà riportato il frontespizio compilato a parte
\end{frontespizio}

\chapter{Frontespizio su Overleaf}

\section{Introduzione}
Molte volte accade che ci sia il bisogno di utilizzare il pacchetto {\sffamily front-th}~\cite{egreg} su \overleaf{}, i pigri studenti del Dipartimento di Informatica di Verona che vogliono comporre la propria tesi non vogliono infatti installare la \emph{TeX Live} sul proprio pc. Questo breve documento ha lo scopo di mostrare i semplici passi per procedere con successo a tale operazione.

\section{Procedura}

Sia che vogliate aggiungere il frontespizio a un progetto già esistente o che vogliate iniziare da zero, il vostro progetto dovrà avere un file \texttt{main.tex}, il documento principale con il testo della tesi. Nei passi che seguono si suppone inoltre che abbiate scelto la classe \texttt{book} per il vostro progetto, se avete fatto altrimenti dovrete adoperarvi con le piccole modifiche necessarie.
La metodologia che applicheremo sarà basata sulla creazione di un documento secondario contenete le direttive per la creazione di un pdf con il frontespizio; tale pdf verrà poi inserito come prima pagina del documento finale tramite una segnaposto sul documento \texttt{main.tex}. Procedete dunque come segue:

\begin{enumerate}
    \item Create tramite il menù di \overleaf{} un nuovo file: \texttt{frontespizio.tex}. 
    \item In questo file andrà inserito un ambiente \texttt{frontespizio} contenente le direttive specifiche del pacchetto stesso~\cite{egreg} con un wrap in un \emph{document}, come spiegato in via ufficiale da \overleaf{}~\cite{frontespizio-overleaf} e riportato a seguito:
\begin{verbatim}
\documentclass[a4paper,titlepage]{book}
\usepackage{frontespizio}

\begin{document}
    \begin{frontespizio}
        \Universita{Verona}
        \Dipartimento{Informatica}
        \Corso[Laurea Magistrale]{Ingegneria e ...}
        \Titoletto{Guida Rapida in due pagine}
        \Titolo{Come usare il pacchetto f...}
        \Candidato[VR434403]{Marco Crosara}
        \Relatore{Prof.~Tyrion Lannister}
        \Annoaccademico{2019-2020}
    \end{frontespizio}
    \immediate\write18{pdflatex \jobname-frn}
\end{document}
\end{verbatim}
    \item A questo punto è necessario inserire il riferimento al frontespizio sul documento principale \texttt{main.tex}, per farlo è sufficente sostituire il classico comando \texttt{\textbackslash{}maketitle} con il seguente segnaposto:
\begin{verbatim}
\begin{frontespizio}

\end{frontespizio}
\end{verbatim}
    Anche su \texttt{main.tex} bisogna inoltre includere, come abbiamo fatto nel caso di \texttt{frontespizio.tex}, il pacchetto necessario:
\begin{verbatim}
\usepackage{frontespizio}
\end{verbatim}    
    \item Infine non ci rimane altro che procedere alla compilazione dei documenti e alla generazione del file di output, seguendo rigorosamente l'ordine sarà quindi necessario:
    \begin{enumerate}
        \item Spostarsi sul documento \texttt{frontespizio.tex} e cliccare su \emph{Recompile}. In tal modo verrà quindi generato il file \texttt{jobname-frn.pdf} relativo al frontespizio (non sarà visibile sull'esplora risorse di Overleaf ma sarà comunque presente nel server).
        \item Spostarsi sul documento principale \texttt{main.tex} e cliccare su \emph{Recompile}. Generando quindi il documento finale che preconcatena il frontespizio \texttt{jobname-frn.pdf} al documento pricipale.
    \end{enumerate}
\end{enumerate}

\'E possibile procedere anche senza l'utilizzo di un file di supporto aggiuntivo \texttt{frontespizio.tex}, integrando direttamente l'ambiente frontespizio sul documento principale \texttt{main.tex}, tuttavia per una pulita struttura del progetto qui è stata presentata la procedura facente uso di un file separato.

\section{Note}

Quando si accederà in futuro al progetto su Overleaf, in mancanza di cookie, può accadere che compilando il documento \texttt{main.tex} non venga generato il frontespizio ma una pagina vuota, questo perchè è necessario procedere nuovamente alla generazione del frontespizio (\texttt{jobname-frn.pdf}). Per sistemare è sufficiente ripetere il punto 4 della procedura, rigenerando quindi prima il frontespizio e procedendo solo poi alla compilazione del documento completo.

% https://it.overleaf.com/learn/latex/Questions/How_do_I_use_the_frontespizio_package%3F

\section{Conclusione}
Questa guida, nella sua semplicità, non vuole essere più che un aiuto agli studenti e un ringraziamento al Maestro di \LaTeX{}.

\bibliographystyle{plain}
\bibliography{references}
\end{document}
